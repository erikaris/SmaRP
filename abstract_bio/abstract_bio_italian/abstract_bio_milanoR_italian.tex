\documentclass[]{article}

%for italian accents
\usepackage[T1]{fontenc}
\usepackage[utf8]{inputenc}
\usepackage[italian]{babel}
%for links
\usepackage{hyperref}

%opening
\title{Condividere il proprio R development:\\
	l'esempio di SmaRP:\\
	Smart Retirement Planning}
\author{Francesca Vitalini}
\date{\today}

\begin{document}

\maketitle

\section*{\normalsize{Abstract}}
Smart Retirement Planning (\textbf{SmaRP}) è un'iniziativa di \href{https://mirai-solutions.ch/}{Mirai Solutions} per supportare le persone che lavorano in Svizzera nelle decisioni riguardanti il proprio fondo pensione: fornisce un'interfaccia utente parametrizzabile ma intuitiva, che riflette la complessità del \href{ttps://en.wikipedia.org/wiki/Pension_system_in_Switzerland}{sistema pensionistico Svizzero} su cui SmaRP si basa.\\

Da un punto di vista tecnico, SmaRP è un'applicazione web sviluppata in \href{https://shiny.rstudio.com/}{R Shiny} ed organizzata in forma di pacchetto R. 
In questa presentazione useremo SmaRP come esempio per mettere in evidenza gli aspetti più interessanti dello sviluppo di applicazioni web in R Shiny. In particolare, sottolineeremo come strutturare una web app in R, come estendere il framework di R Shiny per una migliore interfaccia grafica e come generare un documento riassuntivo tramite la app.

\section*{\normalsize{Biografia}}
Francesca lavora come Solutions Consultant per \href{https://mirai-solutions.ch/}{Mirai Solutions}, una piccola società di consulsulenza basata a Zurigo e specializzata in data science e data analytics per il settore finanzario. In aggiunta al suo ruolo di consulente, Francesca si occupa anche di insegnamento e divulgazione di R. Nel suo tempo libero Francesca partecipa ed organizza eventi per supportare donne nel data science.



\end{document}
