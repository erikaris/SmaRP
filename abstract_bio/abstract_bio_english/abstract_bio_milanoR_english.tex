\documentclass[]{article}

%for italian accents
\usepackage[T1]{fontenc}
\usepackage[utf8]{inputenc}
\usepackage[italian]{babel}
%for links
\usepackage{hyperref}

%opening
\title{Share your R development:\\
	the example of SmaRP:\\
	Smart Retirement Planning}
\author{Francesca Vitalini}
\date{\today}

\begin{document}

\maketitle

\section*{\normalsize{Abstract}}
Smart Retirement Planning (\textbf{SmaRP}) is an initiative of \href{https://mirai-solutions.ch/}{Mirai Solutions} designed to guide people working in Switzerland towards a strategic decision-making process for their retirement: it provides a parametrisable yet intuitive user interface, which reflects the complexity of the \href{ttps://en.wikipedia.org/wiki/Pension_system_in_Switzerland}{Swiss pension system} on which it is based.\\

From a technical perspective, SmaRP is a web application implemented in \href{https://shiny.rstudio.com/}{R Shiny} and organized as an R package. 
In this presentation, using SmaRP as an example, we will identify the most interesting aspects of web apps development in R Shiny. In particular, we will highlight how to structure a wb app in R, how to extend the R Shiny framework for a nicer graphic interface, and how to generate a report through the app.

\section*{\normalsize{Biography}}
Francesca is a Solutions Consultant for \href{https://mirai-solutions.ch/}{Mirai Solutions}, a small data science and data analytics consulting company based in Z\"{u}rich and specialized in the financial sector. In addition to her role as a consultant, Francesca teaches R and is involved outreach activities. In her free time, Francesca participates and organizes events that support women in data science.


\end{document}
